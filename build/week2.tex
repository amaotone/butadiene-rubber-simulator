\documentclass[a4paper,titlepage]{article}
\usepackage{ascmac}
\usepackage{amsmath,amssymb}
\usepackage{siunitx}
\sisetup{group-separator = {,}}
\usepackage[dvipdfmx]{graphicx}

\usepackage{listings}
\usepackage{color}

\lstset{
language={Ruby},
basicstyle={\small\ttfamily},
identifierstyle={\small},
commentstyle={\small\ttfamily \color[rgb]{0,0.5,0}},
keywordstyle={\small\ttfamily\bfseries \color[rgb]{1,0,0}},
ndkeywordstyle={\small},
stringstyle={\small\ttfamily \color[rgb]{0,0,1}},
frame={tb},
breaklines=true,
columns=[l]{fullflexible},
numbers=left,
xrightmargin=0zw,
xleftmargin=3zw,
numberstyle={\scriptsize},
stepnumber=1,
numbersep=1zw,
morecomment=[l]{//}
}

\begin{document}
  \title{Process System Engineering \#2}
  \author{\#03150796 Amane Suzuki}
  \date{}
  \maketitle

  \section{Theory}
  Given reaction rate constant $k$ as a function of temperature.
  Constant temperature condition therefore $k$ is common in all reactor.

  mole balance on reactor 1, with flow rate $v$, reactor size $V$,
  and monomar concentration $C_n$,
  \begin{align}
    C_0v - kC_1V &= C_1v, \\
    \frac{C_1}{C_0} &= \frac{v}{v+kV},
  \end{align}

  mole balance on reactor 2,
  \begin{align}
    C_1v - kC_2V &= C_2v, \\
    \frac{C_2}{C_0} = \frac{C_2}{C_1}\frac{C_1}{C_0} &= \left(\frac{v}{v+kV}\right)^2,
  \end{align}

  therefore,
  \begin{align}
    \frac{C_n}{C_0} = \left(\frac{v}{v+kV}\right)^n.
  \end{align}

  In terms of weight fraction of monomar $\gamma_n$,
  \begin{align}
    \frac{\gamma_n}{\gamma_0} = \left(\frac{v}{v+kV}\right)^n,
  \end{align}

  moreover,
  \begin{align}
    1-\zeta = \frac{\gamma_N}{\gamma_0} &= \left(\frac{v}{v+kV}\right)^N, \\
    \frac{v}{v+kV} &= (1- \zeta)^\frac{1}{N}, \label{eq:prop} \\
  \end{align}

  therefore,
  \begin{align}
    \gamma_n &= (1-\zeta)^\frac{n}{N}\gamma_0.
  \end{align}

  \section{Algorithm}
  Power consumption $P$ of each reactor is defined by $\gamma_{in}$, $\gamma_{out}$,
  and invariable conditions (e.g. $\rho$, $\Delta H$, etc.)

  therefore the structure of the source program is,
  \begin{screen}
    \begin{verbatim}
class Plant
  initialize(N, gamma_0, T2)
  calc()
  reactor(gamma_in, gamma_out)
end

plant = Plant.new(N, gamma_0, T2)
plant.calc()\end{verbatim}
  \end{screen}

  \newpage

  In \verb/initialize()/ function, we setup
  \begin{itemize}
    \item flow rate $v$
    \item reactor size $V$
    \item diameter of reactor $D$
    \item height of reactor $H$
  \end{itemize}

  based on the method of Home Task \#1.

  Given reactor size $V$ is calculated by,
  \begin{align}
    V = \frac{v\tau}{N},
  \end{align}

  In \verb/reactor()/ function, we calculate
  \begin{itemize}
    \item viscosity $\mu$
    \item heat transfer rate $h$
    \item dimentionless numbers $Pr, Nu, Re$
    \item revolution number $n$
    \item power consumption $P$
  \end{itemize}

  In \verb/calc()/ function,
  \begin{enumerate}
    \item calculate $\gamma_n$ by equation (\ref{eq:prop})
    \item call \verb/reactor(/$\gamma_{n-1}$\verb/,/$\gamma_n$\verb/)/ from $n=1$ to $n=N$
    \item output total power consumption $P_{tot}$
  \end{enumerate}

  \newpage

  \section{Result}

  \subsection*{(1)$N=1, \gamma_0=0.05$}
  Use listing \ref{code:normal},
  \begin{screen}
    \begin{verbatim}
$ ruby plant.rb
input n[-], gamma_0[wt%], T2[K]
1
0.05
258
Conditions:
(N, gamma_0, T2)=(1, 0.05, 258)
Reactor Size
V = 514.706[m3]
D = 7.959[m]
H = 10.346[m]
Result:
#1
Re = 6893.441
n = 0.316[rps]
P = 32745.985[W]
Total:
Ptot = 32745.985[W]\end{verbatim}
  \end{screen}

  \begin{table}[htbp]
    \centering
    \begin{tabular}{cc}\hline
      volume & $\SI{514.706}{\cubic\meter}$ \\
      diameter & $\SI{7.959}{\meter}$ \\
      height & $\SI{10.346}{\meter}$ \\
      total power consumption & $\SI{32745.985}{\watt}$ \\ \hline
    \end{tabular}
    \caption{Result in $(N, \gamma_0, T_2) = (1, 0.05, 258)$}
  \end{table}

\newpage

  \subsection*{(2)$N=3, \gamma_0=0.04$}
  Use listing \ref{code:normal},
  \begin{screen}
    \begin{verbatim}
$ ruby plant.rb
input n[-], gamma_0[wt%], T2[K]
3
0.04
258
Conditions:
(N, gamma_0, T2)=(3, 0.04, 258)
Reactor Size:
V = 214.461[m3]
D = 5.944[m]
H = 7.728[m]
Results:
#1
Re = 9653.252
n = 0.082[rps]
P = 120.778[W]
#2
Re = 3062.065
n = 0.103[rps]
P = 334.018[W]
#3
Re = 1378.443
n = 0.092[rps]
P = 293.009[W]
Total:
Ptot = 747.805[W]\end{verbatim}
  \end{screen}

  \begin{table}[htbp]
    \centering
    \begin{tabular}{cc}\hline
      volume & $\SI{214.461}{\cubic\meter}$ \\
      diameter & $\SI{5.944}{\meter}$ \\
      height & $\SI{7.728}{\meter}$ \\
      total power consumption & $\SI{747.805}{\watt}$ \\ \hline
    \end{tabular}
    \caption{Result in $(N, \gamma_0, T_2) = (3, 0.04, 258)$}
  \end{table}

  \newpage

  \subsection*{(3)$N=1 \mathrm{to} 5, \gamma_0 = 0.02 \mathrm{to} 0.10$}
  Use listing \ref{code:advanced},

  \begin{screen}
    \begin{verbatim}
$ ruby plant-advanced.rb
2.47	0.31	0.12	0.06	0.04
6.92	0.87	0.33	0.18	0.11
15.9	2.01	0.75	0.41	0.26
32.75	4.13	1.54	0.84	0.54
62.92	7.94	2.96	1.61	1.04
115.28	14.55	5.42	2.95	1.91
203.95	25.75	9.59	5.22	3.38
351.32	44.35	16.52	8.99	5.82
592.59	74.81	27.87	15.16	9.81\end{verbatim}
  \end{screen}

  \begin{table}[htbp]
    \centering
    \begin{tabular}{l|lllll}\hline
      & \multicolumn{5}{c}{$N$} \\
      $\gamma_0$ & 1 & 2 & 3 & 4 & 5 \\ \hline
      0.02 & 2.47 & 0.31 & 0.12 & 0.06 & 0.04 \\
      0.03 & 6.92 & 0.87 & 0.33 & 0.18 & 0.11 \\
      0.04 & 15.9 & 2.01 & 0.75 & 0.41 & 0.26 \\
      0.05 & 32.75 & 4.13 & 1.54 & 0.84 & 0.54 \\
      0.06 & 62.92 & 7.94 & 2.96 & 1.61 & 1.04 \\
      0.07 & 115.28 & 14.55 & 5.42 & 2.95 & 1.91 \\
      0.08 & 203.95 & 25.75 & 9.59 & 5.22 & 3.38 \\
      0.09 & 351.32 & 44.35 & 16.52 & 8.99 & 5.82 \\
      0.10 & 592.59 & 74.81 & 27.87 & 15.16 & 9.81 \\ \hline
    \end{tabular}
    \caption{Power Consumption $P [\si{\kilo\watt}]$ in each condition}
  \end{table}

  \newpage

  \section{Source Program}
  \lstinputlisting[caption=plant.rb, label=code:normal]{code/reactor.rb}

  \newpage

  \lstinputlisting[caption=plant-advanced.rb, label=code:advanced]{code/reactor2.rb}
\end{document}
